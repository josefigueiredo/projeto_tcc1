
%\chapter{Introdução}


\chapter{Introdução}

A introdução apresenta os objetivos do trabalho, bem como as razões de sua elaboração. Tem caráter didático de apresentação.

Deve abordar:
\begin{enumerate}[noitemsep,nosep,labelindent=\parindent,leftmargin=*,label={\alph*}) ] 
	\item o problema de pesquisa, proposto de forma clara e objetiva;
	\item os objetivos, delimitando o que se pretende fazer;
	\item a justificativa, destacando a importância do estudo;
	\item apresentar as definições e conceitos necessários para a compreensão do estudo;
	\item apresentar a forma como está estruturado o trabalho e o que contém cada uma de suas partes.
\end{enumerate}

O desenvolvimento é a demonstração lógica de todo o trabalho, detalha a pesquisa ou o estudo realizado. Explica, discute e demonstra a pertinência das teorias utilizadas na exposição e resolução do problema. 

O desenvolvimento pode ser subdivido em seções e subseções com nomenclaturas definidas pelo autor conforme conteúdo apresentado. 

Regras de apresentação da Capa


\section{SEÇÃO SECUNDÁRIA}

A ABNT indica a elaboração de uma lista de ilustrações com todos os itens arrolados e designados por seu nome específico, conforme a ordem que aparecem no texto (Figura 1, Fotografia 1, Gráfico 1, Quadro 1, entre outros). Também recomenda, quando necessário, a elaboração de lista própria para cada tipo de ilustração. No entanto, não determina um número mínimo de ilustrações para tal lista específica.

Este trabalho foi adaptado de um documento público disponibilizado pela UDESC (Universidade do Estado de Santa Catarina). Uma das diferenças que este trabalho contém é a citação segundo o IEEE (\textit{Institute of Electrical and Electronic Engineers}). Logo, a adaptação adequa-se ao projeto de pesquisa que pode ser apresentado na disciplina Tópicos Especiais em Telecomunicações II.




\subsection{Seção terciária}

Nesse caso, a BU Udesc estabelece a elaboração de listas específicas para cada tipo de ilustração somente quando existirem muitos itens de cada tipo: cinco (5) ou mais (mais do que cinco desenhos, gráficos etc.). Caso contrário, elabora-se uma única lista, denominada “Lista de ilustrações” com os elementos ordenados conforme aparecem no texto, nominando-os “Figura” e, portanto, não diferenciando fotografia, gráfico, quadro e outros.

O vídeo fornece uma maneira poderosa de ajudá-lo a provar seu argumento. Ao clicar em Vídeo Online, você pode colar o código de inserção do vídeo que deseja adicionar.





