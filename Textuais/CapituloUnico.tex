%deixar chapter vazio - exatamente como está. Preencher o documento apenaas apartir de section.
\chapter*[]{}

\section{Tema}

Simulação Computacional

\subsection{Delimitação do tema}

Detecção de colisão entre objetos em simulações computacionais

\section{Problema}

 Estudar e avaliar os mecanismos para detecção de colisão em uma simulação computacional que busca apresentar objetos de forma mais realística possível.


\section{Objetivos}

\subsection{Objetivo Geral}

\subsection{Objetivos Específicos}
\begin{enumerate}
\item Conhecer as principais etapas de uma simulação física computacional.
\item Avaliar o funcionamento dos algoritmos sort and sweep e octree na fase de detecção de possíveis colisões em uma etapa de simulação.
\item Avaliar o funcionamento de algoritmos para detecção de colisão entre objetos de geometrias específicas.
\item Avaliar o método Impulse para a resolução de colisão entre dois objetos em uma etapa de simulação.
\end{enumerate}

\section{Justificativa}

Segundo \citeonline{bourg2013physics}, “a detecção de colisão é um problema de geometria computacional que determina se ocorreu, e onde ocorreu, uma ou mais colisões entre objetos” da simulação. Com base nisto, podemos afirmar que existem diversos algoritmos para detecção de colisão entre objetos de geometrias específicas, e também diferentes técnicas para otimizar esta detecção tornando-a mais eficiente, rápida e robusta.

\citeonline{ericson2004real} afirma que apesar de hoje em dia os computadores serem extremamente rápidos, a detecção de colisão continua sendo um problema fundamental. Isso ocorre porque jogos e outras simulações são frequentemente construídos em um ambiente 3D, com centenas de milhares de polígonos, exigindo assim, algoritmos e estruturas de dados mais sofisticados para operar, em tempo real, com conjuntos de dados dessa magnitude.

Por ser uma área ainda em expansão, pouco explorada e conhecida, este estudo se justifica pois novos métodos poderão surgir, havendo possibilidades para vários trabalhos de pesquisa e desenvolvimento.

\section{Referencial Teórico}


\subsection{Simulação Computacional}

Segundo \citeonline{duran2018computer} A simulação computacional é uma técnica amplamente utilizada para o estudo de
sistemas complexos que por diversos fatores não podem ser facilmente
reproduzidos devido a custos elevados , ou pelo tempo necessário para que se
realize os experimentos para a obtenção de resultados precisos.

A simulação,permite que um pesquisador avalie diversos cenários diferentes para sua
pesquisa apenas alterando suas variáveis de controle reduzindo a sim os custos
e os riscos de um experimento físico. Essa técnica pode ser aplicada em
diversas áreas como biologia, engenharia, física, química entre outras.

Para construir uma simulação, torna-se necessária a representação matemática ou
computacional do comportamento do sistema a ser simulado. Isso inclue mas não
se limita a variáveis de controle que podem alterar o ambiente simulado, e os
objetos que serão simulados durante algumas etapas, ou por um período de tempo.

A simulação é uma poderosa ferramenta que contribue na descoberta de novas
tecnologias, ou no aprimoramento das existentes.


\subsubsection{Representando Formas Geométricas Computacionalmente}
Na computação, não existe maneira correta ou incorreta de representar a
geometria. A forma usada dependerá  do nível de detalhamento desejado. É possível fazer esta representação de diversas maneiras, por exemplo: forma vetorial, paramétrica ou de forma poligonal.

Neste estudo será adotada a representação vetorial, devido a sua simplicidade e fácil compreensão. O trabalho será desenvolvido com 3 geometrias principais, que são: esferas, caixas e poliedros.

Para representação dessas formas geométricas, são utilizados pontos de coordenadas cartesianas, definidos em uma estrutura de vetor tridimensional. O código a seguir demonstra essa estrutura.

\begin{lstlisting}[frame=single,caption=Exemplo de vetor 3d\label{code:vec3d_1}]
class vector3d{
float x;
float y;
float z;

vector3d(float x, float y, float z){
this->x=x;
this->y=y;
this->z=z;
}
};
\end{lstlisting}

Onde x representa o eixo que vai da esquerda para a direita;
y representa o eixo  perpendicular ao eixo x;
E z é o eixo que indica a profundidade apontando para cima.


Lembrando também que os vetores tem diversas operações relacionadas como soma,
subtração, multiplicação, divisão, Comprimento ou módulo, produto escalar,
produto cruzado e produto triplo escalar que são relevantes para este estudo.

\subsubsection{Esferas}

As esferas são simples de representar computacionalmente, e também
incrivelmente fáceis de se computar. Para representarmos uma esfera, tudo o que
precisamos é de um ponto que representará o centro da esfera, e um raio, que
indica o quanto a esfera se estende em todas as direções a partir do centro.

\begin{lstlisting}[frame=single,caption=Representação de esfera\label{code:sphere3d}]
class sphere3d {
public:
vector3d center;
float radius;
sphere3d(const vector3d& center, float radius)
{
this->center=center;
this->radius=radius;
}
};

sphere3d s({0.0f, 0.0f, 0.0f}, 5.0f);
\end{lstlisting}

A sim temos uma esfera cujo o centro está posicionado exatamente na origem de
nosso sistema cartesiano, e que se estende por um raio de 5 em todas as
direções.

\subsubsection{Caixas}

As caixas são trivialmente representáveis de diversas formas. Com elas podemos
representar formas cuboides facilmente. Ela normalmente é representada por dois
vetores, onde um indica o canto inferior esquerdo, e o outro o canto superior
direito. No livro Real Time Collision Detection, Christer Ericson demonstra
várias maneiras de se representar AABBS, que nada mais são que caixas.
Outras formas populares de se representar caixas são:
Ponto central mais meias larguras que se estenderão do centro até as bordas da caixa.
Ou a forma que usaremos que consiste em um ponto mínimo e um vetor
representando a largura, altura e comprimento da caixa respectivamente.
A seguir as três formas descritas são representadas. Ambas fornecem vantagens e
desvantagens dependendo do contexto em que forem utilizadas.

Caixa representada pelo ponto mínimo e máximo:

\begin{lstlisting}[frame=single,caption=Representação de caixa\label{code:box3d_1}]
class box3d_1{
vector3d min;
vector3d max;
};
\end{lstlisting}

Caixa com meia larguras

\begin{lstlisting}[frame=single,caption=Exemplo de caixa 2\label{code:box3d_2}]
class box3d_2
{
vector3d center;
vector3d radius;
};
\end{lstlisting}

Caixa com ponto mínimo e medidas completas:

\begin{lstlisting}[frame=single,caption=Exemplo de caixa 3\label{code:box3d_3}]
class box3d_3 {
vector3d min;
vector3d measures;
};
\end{lstlisting}

Todas elas servem para representar a mesma coisa, mas podem ser utilizadas em
contextos diferentes dependendo o requisito. A ~\ref{code:box3d_1} é excelente para aabbs
porque já tem seus limites bem definidos e pode ser usada mais facilmente para
sucessivos testes. Porém tem a desvantagem de precisar fazer uma atualização
extra quando precisar ser movimentada.
No entanto, a ~\ref{code:box3d_2} tem a vantagem de precisar atualizar apenas o ponto
central, porém, quando necessário, terá as operações para computar seus pontos
mínimos e máximos. Este tipo de estrutura também é conhecida como esfera de 3
raios. Em octrees ela é a representação preferida para a representação de nós
pois é fácil de representar e de se computar.
A ~\ref{code:box3d_3} Lembra muito a criação de uma janela na computação gráfica. Ela tem seu
ponto mínimo, e o comprimento de suas arestas até o ponto máximo.

Os 3 exemplos abaixo criam 3 caixas exatamente com os mesmos pontos finais.

\begin{lstlisting}[frame=single,caption=Código de exemplo\label{example_box3d}]
box3d_1 b1({10,10,10}, {20,20,20});
box3d_2 b2({15,15,15}, {5,5,5});
box3d_3 b3({10,10,10}, {10,10,10});
\end{lstlisting}

Todas as 3 representações terão seus limites entre 10 e 20.


\subsubsection{Poliedros}

Os poliedros nada mais são que uma nuvem de pontos que representam uma
geometria específica. Os poliedros certamente são a maneira mais precisa de
representar geometria, porém eles são uma faca de dois gumes. Quanto mais
detalhado o poliedro, mais caro será para computar colisões, rotações e
translações. Existem dois tipos de poliedros. Poliedros convexos e poliedros
côncavos. Poliedros convexos são poliedros que são coplanares ou que possuem
suas faces voltadas para fora. Por outro lado, poliedros côncavos possuem
pelo menos uma de suas faces voltada para o interior.
Neste trabalho não usaremos poliedros pois eles utilizam algoritmos mais
sofisticados para operarem sobre nuvens de pontos para detectar colisões.
Alguns dos algoritmos mais famosos para isto incluem o
GJK(Gilbert Johnson Keerthi), ou o Teorema dos eixos separadores, na sigla em
inglês (SAT).

\subsection{ Estruturas para delimitação de volumes}
Em uma simulação,temos os mais variados objetos interagindo uns com os outros.
Como por exemplo, uma esfera colidindo contra um triângulo representado de
forma poligonal. Na simulação, teremos diversos objetos e precisamos testar a
colisão entre um dos objetos contra todos os outros com seus algoritmos e
métodos específicos, o que pode ser computacionalmente caro e custoso.

É para resolver esse problema de executar algoritmos caros a todo instante que
surgiram os volumes delimitadores. É escolhida uma forma arbitrária para
representar um volume delimitador para encapsular a geometria específica.

Existem vários tipos de volume delimitadores como esferas, retângulos, e cascas
convexas. Todas elas tem suas desvantagens e vantagens como por exemplo, um
volume delimitador como uma esfera é extremamente simples e rápido de se
calcular, porém, ela é autamente imprecisa para detectar colisões gerando
diversos falsos positivos.

A forma mais precisa é uma casca convexa, porém, ela é cara de se computar.
Então um retângulo fornece um bom equilíbrio entre precisão e velocidade sendo
a sim, a forma preferida de volume delimitador.

Depois de escolhida a forma do volume delimitador que aqui foi escolhida o
retângulo, surge outro problema. Que tipo de volume delimitador queremos? Um
AABB, ou um OBB? Novamente, ambos tem suas vantagens e desvantagens que são
explicadas a seguir:

\subsubsection{AABB}
Aligned Axis Bounding Box é uma caixa retangular alinhada aos três eixos do
sistema de coordenadas, por tanto, ela não pode sofrer rotações. Então, caso a
geometria contida por este volume delimitador sofra uma rotação, o volume
delimitador deve ser reconstruído para acomodar a geometria rotacionada. Isso
pode ser especialmente custoso de se computar em poliedros com muitos vértices.

No entanto, a vantagem é que o teste de colisão é eficiente e relativamente
simples de se computar.

\begin{lstlisting}[frame=single,caption=Exemplo de AABB\label{codigo1}]
class AABB
{
public:
vector3d min;
vector3d max;
GeometricShape* shape;
};

sphere3d* s=new sphere3d({5, 5, 5}, 5);
AABB* aabb=new AABB(s);
\end{lstlisting}

A bounding box desta esfera será 2 vetores que envolverão completamente em um cubo.
Seus vetores respectivamente serão:
min={0,0,0}
max={10,10,10}


\subsubsection{OBB}

Um Oriented Bounding box tem a mesma função de um AABB, porém, ele não está
preso a nenhum eixo e é invariável a rotação dos objetos pois ele também é
capaz de rotacionar acompanhando a rotação do objeto contido. Ele é
representado de forma diferente de um aabb com um ponto central, as meia
medidas para os eixos, e uma matriz de rotação para servir como orientação do
obb. É possível representar a matriz de rotação como ângulos de Euler para
economizar memória, porém, mais tarde ao atualizar a rotação, terá que
converter os ângulos em uma matriz. Como Christer Ericson comenta em seu livro,
Obbs e conversões de ângulos é surpreendentemente caro e complicado.

Como tudo na computação tem seus prós e contras, obbs são geralmente mais
precisos que AABBS, porém, são mais caros de se computar, e também mais
difíceis de se implementar.


Algumas representações típicas de obbs se seguem:

\begin{lstlisting}[frame=single,caption=Código de exemplo de OBB\label{code:obb_1}]
class obb_1
{
public:
vector3d center;
vector3d measures;
matrix3x3 orientation;
};
\end{lstlisting}

\begin{lstlisting}[frame=single,caption=Exemplo de OBB\label{code:obb_2}]
class obb_2
{
public:
vector3d center;
vector3d measures;
vector3d angles;
};
\end{lstlisting}

Para realizar um teste de sobreposição obb obb é necessário utilizar o teorema
dos eixos separadores que diz que se existir ao menos um plano que divida os
dois objetos, eles não estão colidindo.



\subsubsection{ Corpo Rígido}
\subsubsection{ O que é um corpo rígido?}

Segundo Bourg e Bywalec (2013, tradução nossa) Um corpo rígido é um conjunto de partículas  que permanecem a distâncias fixas umas das outras  sem nenhum tipo de translação ou rotação entre elas. Em outras palavras,  um corpo rígido não mudará de forma enquanto se move, ou sua deformação é tão insignificante que pode ser desprezada.

Um corpo rígido é  frequentemente representado como tendo uma posição, uma orientação  e  dimensões para representar seu volume. Estes são requisitos mínimos para a  sua definição, porém, dependendo do nível de precisão desejada, pode-se adicionar mais atributos como tensores de inércia, velocidades lineares e angulares, coeficiente de restituição, e o que mais for necessário para que se atinja os objetivos desejados.

\subsubsection{ Diferença entre corpo dinâmico e corpo estático}

Existem 3 classes principais de corpos que podem ser representadas em uma simulação:
Corpos rígidos, que não se deformam  enquanto se movem, ou caso se deformem, são capazes de voltar a sua forma original;


Corpos macios, que podem deformar enquanto se movem ou por ação de forças ou eventos externos;

Ou corpos estáticos, que nada mais é que uma sub-classificação de um corpo rígido ou um corpo macio,  mas com a ausência de massa, ficando a sim, estacionário na posição em que se
encontra...

\subsubsection{"Representação de um Corpo Rígido}

Para fins de exemplos, a seguinte representação de corpo rígido será adotada:

\begin{lstlisting}[frame=single,caption=Exemplo de corpo rígido\label{code:RigidBody}]
class RigidBody
{
public:
float mass;//Massa do objeto...
float restitution;//Coeficiente de restituicao...
vector3d lastPosition;//Ultima posicao do objeto...
vector3d position;//Posicao atual do objeto...
vector3d velocity;//Velocidade do objeto...
vector3d forces;//Forcas que estao atuando sobre o objeto...
AABB* aabb;//Forma do objeto...
};
\end{lstlisting}

\subsection{Detecção de Colisão}
Segundo \citeonline{bourg2013physics}, a detecção de colisão é um problema de geometria computacional que
determina se e onde, dois ou mais objetos colidiram.

Uma colisão pode ser detectada de diversas maneiras dependendo da natureza dos
objetos envolvidos como esferas e caixas.

Para algumas situações, uma colisão
pode ser detectada se a distância entre os dois objetos está a baixo de uma
determinada tolerância. Em outros casos, os objetos podem estar se sobrepondo
em um ou mais pontos.

Após uma colisão ser detectada, informações devem ser coletadas para a
resolução da colisão posteriormente.

Quais informações coletar depende muito da natureza do simulador. Um simulador
de física por exemplo pode precisar do ponto, ou pontos de contato que fizeram
os objetos colidirem, além de sua velocidade relativa no instante da colisão.


Outras informações importantes para se coletar durante a detecção de colisão
inclue o vetor normal, que é um vetor unitário que apontará na direção em que a
colisão aconteceu.


E por último, a profundidade de sobreposição, ou de penetração dos objetos.
Esta informação é necessária para separar os objetos para que eles apenas se
toquem, e não se sobreponham.


\subsection{Fases da Detecção de Colisão}

A detecção de colisão é dividida em duas áreas principais que são:

\subsubsection{Fase Ampla}
A fase ampla é responsável por determinar rapidamente pares de objetos que
possivelmente estejam colidindo, e descartar aqueles objetos que não estão.

Nesta fase, os AABBS das formas geométricas são testados pois possuem
algoritmos simplificados e eficientes para esta tarefa.

Quando uma colisão entre dois AABBS é detectada, a sua geometria contida pode
não estar colidindo com a geometria do segundo AABB. Então estes dois objetos
são agrupados para uma análise mais detalhada em uma fase posterior.

\subsubsection{Fase Estreita}
A fase estreita é responsável pelo trabalho mais detalhado e preciso da
detecção de colisão.

É nesta fase onde os algoritmos e técnicas específicas são
aplicadas para determinar se de fato, os pares de objetos suspeitos de colisão
estão colidindo.

Caso estejam, informações como normal, ponto de contato e profundidade de
penetração devem ser calculados para uso posterior.

\subsection{Algoritmos para Fase Ampla}
Um problema da detecção de colisão é que em uma simulação podem existir muitos
objetos. Alguns distantes, e outros próximos.

Quando pegamos um objeto de forma aleatória deste conjunto de objetos queremos saber com quais outros objetos
existe a chance de colisão.
Em uma situação desta natureza, a complexidade dos testes normalmente é de $\mathcal{O}(N^2)$, o que
computacionalmente é muito dispendioso, porque os objetos mais distantes
espacialmente não precisariam ser testados.

Para acelerar este processo, existem diversas estruturas de dados e técnicas de compartimentação espacial.
A lógica destas técnicas é agrupar os objetos que estão próximos uns dos outros
e interromper a busca quando o restante dos objetos estão fora de alcance.

Algumas das estruturas mais populares atualmente são as árvores octrees, grades
uniformes e a divisão espacial por hash.

\subsubsection{Octrees}

Uma octree é uma estrutura de dados hierárquica que organiza um espaço
tridimensional em células menores recursivamente.
Dado um nó raiz, este nó é dividido em 8  filhos, que também podem ser
subdivididos em mais 8 filhos até que um critério de parada seja satisfeito.
Este critério pode ser uma profundidade máxima para a árvore, ou o tamanho do
nó que será dividido atingiu um limite mínimo.

Os objetos adicionados nesta estrutura tentarão ser adicionados no menor nó que
seja capaz de conter todo o volume do objeto. Com esta compartimentação
espacial, os objetos só precisam verificar a possibilidade de colisão com seus
irmãos, e com os nós mais profundos acelerando assim o processo.

\subsubsection{Ordenar e Varrer}

Na detecção de colisão, existe uma categoria de algoritmos denominada de
classificação e varredura. Estes algoritmos tem como objetivo determinar grupos
de objetos que estão colidindo uns com os outros.

Esta categoria de algoritmos é dividida em duas etapas principais:

Classificação: Classifica os objetos seguindo um critério específico. Este
critério pode ser ordenar os objetos ao longo de um determinado eixo.

Varredura: Varre os objetos classificados detectando interseções entre os os
objetos que estão na projeção do intervalo especificado.

Estes algoritmos possuem diversas vantagens de uso como eficiência de memória,
fácil implementação, e eficiência computacional.

Neste trabalho, o algoritmo sort and sweep (Ordenar e varrer) explicado
detalhadamente em \citeonline{ericson2004real} será utilizado para o desenvolvimento.

O algoritmo começa ordenando o vetor de entrada por um dos eixos X, Y ou Z., em
seguida, o vetor é percorrido, e em cada iteração os objetos são testados
quanto a colisão. Caso o o limite inferior do ~\ref{code:aabb} seja maior do
que o limite superior do AABB testado, o laço interno pode parar com segurança
pois não existe mais risco de colisão com os próximos objetos. Em seguida, a
variância dos eixos são calculados, e o eixo com a maior variância é
selecionado para uma execução futura.

\begin{lstlisting}[frame=single,caption=Exemplo de ordenar e varrer\label{code:SortAndSweep}]
void SortAndSweepAABBArray(void)
{
    // Sort the array on currently selected sorting axis (gSortAxis)
    qsort(gAABBArray, MAX_OBJECTS, sizeof(AABB *), cmpAABBs);
    // Sweep the array for collisions
    float s[3] = { 0.0f, 0.0f, 0.0f }, s2[3] = { 0.0f, 0.0f, 0.0f }, v[3];
    for (int i = 0; i < MAX_OBJECTS; i++) {
          // Determine AABB center point
          Point p = 0.5f * (gAABBArray[i]->min + gAABBArray[i]->max);
          // Update sum and sum2 for computing variance of AABB centers
          for (int c = 0; c < 3; c++) {
                s[c] += p[c];
                s2[c] += p[c] * p[c];
          }
          // Test collisions against all possible overlapping AABBs following current one
          for (int j = i + 1; j < MAX_OBJECTS; j++) {
                // Stop when tested AABBs are beyond the end of current AABB
                if (gAABBArray[j]->min[gSortAxis] > gAABBArray[i]->max[gSortAxis])
                      break;
                if (AABBOverlap(gAABBArray[i], gAABBArray[j]))
                      TestCollision(gAABBArray[i], gAABBArray[j]);
          }
      }
    // Compute variance (less a, for comparison unnecessary, constant factor)
        for (int c = 0; c < 3; c++)
              v[c] = s2[c] - s[c] * s[c] / MAX_OBJECTS;
        // Update axis sorted to be the one with greatest AABB variance
        gSortAxis = 0;
        if (v[1] > v[0]) gSortAxis = 1;
        if (v[2] > v[gSortAxis]) gSortAxis = 2;
    }
\end{lstlisting}

\subsection{Algoritmos de Fase Estreita}

Após  os testes da fase ampla retornarem pares de objetos que possivelmente
estejam colidindo, é a vez da fase estreita ser executada.

Nesta fase, os algoritmos mais custosos e complexos computacionalmente são
executados para determinar com certeza se os objetos estão em colisão.
Caso estejam, pode-se coletar as informações sobre a colisão como normal, ponto
de contato e profundidade de penetração para resolver a colisão mais tarde.

\subsubsection{Esfera contra Esfera}

Para que duas esferas estejam em colisão, a soma de seus raios deve ser menor
ou igual a distância de seus centros segundo \citeonline{ericson2004real}.
A representação da esfera descrita em ~\ref{code:sphere3d} será utilizada.

\begin{lstlisting}[frame=single,caption=Colisão entre esferas\label{code:collisionSphereSphere}]
bool sphereSphere(sphere3d* s1, sphere3d* s2)
{
//Calcular a distancia quadrada entre os centros das esferas...
vector3d v=s1->center-s2->center;
float dist=((v.x*v.x)+(v.y*v.y)+(v.z*v.z));
//Calcula a soma dos raios das esferas ao quadrado...
float sqRadius=s1->radius+s2->radius;
sqRadius=sqRadius*sqRadius;
//Caso a distancia seja menor ou igual a soma dos raios ao quadrado, as esferas estao colidindo...
if(dist<=sqRadius)
{
//Coletar informacoes relevantes aqui...
return true;
}
return false;
}
\end{lstlisting}

Neste exemplo, a soma dos raios foram elevados ao quadrado para realizar o
teste contra a distância também ao quadrado entre os centros das esferas.
Esta é uma prática recomendada pois o cálculo de raiz quadrada pode ser muito
lento para ser realizada como descrito em \citeonline{ericson2004real} e \citeonline{bourg2013physics}.

\subsubsection{Caixa contra Caixa}

Para que duas caixas estejam em colisão, precisa existir sobreposição nos três
eixos testados.
Neste exemplo, a representação de ~\ref{code:box3d_1} será preferida.

\begin{lstlisting}[frame=single,caption=Colisão entre caixas\label{code:collisionBoxBox}]
bool boxBox(box3d* b1, box3d* b2)
{
//Teste no eixo x...
if((b1->min.x>b2->max.x)||(b2->min.x>b1->max.x))
return false;
//Teste no eixo y...
if((b1->min.y>b2->max.y)||(b2->min.y>b1->max.y))
return false;
//Teste no eixo z...
if((b1->min.z>b2->max.z)||(b2->min.z>b1->max.z))
return false;
//Todos os intervalos se sobrepoem, entao as caixas estao colidindo.
//Coletar informacoes aqui...
return true;
}
\end{lstlisting}

Existem outras variações para este mesmo algoritmo descritos com mais detalhes
em \citeonline{ericson2004real}.

\subsubsection{Esfera contra Caixa}

O teste de esfera e caixa pode ser realizado em duas etapas principais:
Primeiro: encontrar o ponto mais próximo da caixa em relação a esfera.
Segundo: Calcular a distância entre o ponto mais próximo e o centro da esfera.
Caso a distância seja menor que o raio da esfera, então existe uma colisão.

\begin{lstlisting}[frame=single,caption=Colisão entre esfera e caixa\label{code:collisionSphereBox}]
bool spherebox(sphere3d* s, box3d* b)
{
vector3d closestPoint;
                       for (int i = 0; i < 3; i++) {
                             float v = s->center[i];
                             if (v < b->min[i]) v = b->min[i]; // v = max(v, b.min[i])
                             if (v > b->max[i]) v = b->max[i]; // v = min(v, b.max[i])
                             closestPoint[i] = v;
                       }
float sqdist=((closestPoint-s->center)*(closestPoint-s->center));
float sqradius=(s->radius*s->radius);
if(sqdist<=sqradius)
{
//Coletar as informacoes...
return true;
}
//Nao esta colidindo...
return false;
}
\end{lstlisting}

Implementações mais sofisticadas podem ser encontradas em \citeonline{ericson2004real}.

\subsubsection{Informações sobre a Colisão}

Se o objetivo for determinar  se os objetos colidem, os testes demonstrados são
suficientes. No entanto, se o objetivo for resolver a colisão informações adicionais devem
ser coletadas durante o processo de detecção de colisão. O processo de coleta
de informações é um problema difícil segundo \citeonline{ericson2004real}.
Quais informações coletar depende muito dos objetivos do sistema. Porém, três
informações são fundamentais:
Ponto de contato: Retorna um ou mais pontos de contato indicando onde os objetos colidiram.
Normal da colisão: Este é um vetor unitário apontando a direção em que os objetos devem ser movidos para que eles parem de colidir.
Profundidade de penetração: É o quanto os objetos estão se sobrepondo. Esta
informação é utilizada juntamente com a normal de colisão para calcular o vetor
de translação mínima (VTM) ou (MTV) em inglês, necessário para separar os dois
objetos.

Cuidados Adicionais devem ser tomados ao se determinar a normal da colisão como
por exemplo, se uma esfera colidir contra uma caixa, e seu centro estiver
dentro da caixa, o vetor normal será um vetor nulo, o que poderá resultar em
comportamentos indesejados na hora de resolver a colisão.

\subsection{Resolução de Colisão}

Após detectar uma colisão entre dois objetos torna-se necessário resolver
a mesma. Em simulações que não requerem um realismo muito grande, apenas a
correção das posições dos objetos torna-se necessária.

Em simulações mais avançadas, como simulações de física, outros métodos podem
ser utilizados como a resolução de colisão por impulso.

Neste método, características físicas dos objetos são levados em consideração
para que os objetos reajam de forma adequada e mais natural.

Outro cuidado deve-se ser levado em conta nesta etapa.
Os objetos em colisão podem ser um objeto estático e um objeto dinâmico, ou
dois objetos dinâmicos. A colisão não deve ser resolvida caso os dois objetos
em colisão sejam estáticos.
Caso um dos objetos seja dinâmico e o outro estático, a resolução da colisão
deve atuar apenas no objeto dinâmico.
Caso os dois objetos sejam dinâmicos, a resolução deve atuar sobre os dois
objetos em direções opostas.

O exemplo de código a seguir exemplifica este processo...


\begin{lstlisting}[frame=single,caption=Exemplo de resolução de colisão\label{code:solveCollision}]
void solveCollision(object* b1, object* b2, vector3d contactPoint, vector3d normal, vector3d depth)
{
//Supondo que b1 seja estatico e b2 dinamico...
if(b1->mass<=0.0f)
{
//Verificar se r2 esta se afastando de r1 utilizando o produto escalar dos vetores...
float dir=dotProduct(r2->velocity, normal);
//Se dir for maior que 0, significa que os objetos estao se afastando um do outro, portanto nao a nada a ser feito.
if(dir>0) return;
//Calcular o vetor de translacao minima se necessario...
if(depth>0)//b2 esta penetrando em b1...
{
//Multiplica o vetor unitario normal pela profundidade...
//Isso nos dara quanto devemos transladar b2 para que ele apenas se apoie em b1...
vector3d mtv=normal*depth;
//move b2 para longe de b1...
b2->Translate(mtv);
}
//Calcular o impulso a ser aplicado em b2...
float j=-(b2->velocity*normal) * (b2->restitution+1)*b2->mass;
//Aplica o impulso na direcao da normal...
vector3d impulse=(j*info.normal);
//Modifica a velocidade...
r1->velocity+=(1 / r1->mass) * impulse;
}
else
{
//Algo semelhante caso os dois objetos sejam dinamicos...
}
}
\end{lstlisting}

Este é apenas um exemplo básico e simplificado. Em simulações complexas muitos
Outros atributos devem ser levados em consideração como tensores de inércia,
velocidade tangencial, velocidades angulares e torques.

Em \citeonline{bourg2013physics} explicações mais detalhadas podem ser encontradas bem como algoritmos e
métodos mais sofisticados.


\section{Metodologia}

A  proposta de desenvolvimento deste projeto é um programa gráfico que
executará simulações de movimento em ambientes 3d, para analisar os mecanismos de detecção de colisão.
Este programa será implementado com algumas leis básicas da física, para que os objetos simulados reajam de forma mais realista,  tornando assim, a simulação mais dinâmica.
O usuário terá a opção de definir as variáveis de seu ambiente de simulação, como limites espaciais cujo os objetos podem percorrer, definir obstáculos, aplicar a força da gravidade,
definir o passo de tempo da simulação, bem como configurar individualmente cada
objeto presente na mesma.
Os objetos simulados terão os seguintes parâmetros configuráveis: massa,
posição, velocidade, coeficiente de restituição e forma. Os parâmetros serão configuráveis por meio de um painel.
O programa será organizado em 3 partes principais:
Tela principal:
Será onde a simulação será desenhada, mostrando ao usuário o andamento da simulação.
A tela principal também terá um menu permitindo o controle de diversos aspectos da simulação.
Configuração do ambiente:
Esta tela será responsável por configurar o ambiente de simulação. Propriedades
como limites do ambiente, passo de tempo, e se a gravidade deve ou não ser
aplicada são exemplos de propriedades configuráveis.
Também nesta tela, terá a lista de objetos que poderão ser controlados.
Engine de simulação:
Esta parte do programa será subdividida em simulação do movimento e detecção de colisão.
Simulação do movimento:
Esta etapa é responsável pelos cálculos que modelam o movimento dos objetos presentes no ambiente da simulação, utilizando para isto as leis de Newton.
No entanto, a  única restrição existente é que os objetos estáticos não devem ser movimentados. Pois, eles modelam obstáculos ou a geografia do ambiente.
Detecção de colisão:
A detecção de colisão é a etapa responsável por aplicar algoritmos específicos nos objetos a fim de determinar se existe uma sobreposição, ou se  a distância entre 2 objetos é inferior a uma determinada tolerância.
Esta etapa também é responsável por resolver a colisão caso exista.
Modelagem preliminar do simulador:

Nesta cessão, a modelagem preliminar da engine de simulação será discutida apresentando seus principais componentes.

\begin{lstlisting}[frame=single,caption=Código de exemplo\label{codigo1}]
classe vetor3d
{
eixo_x
eixo_y
eixo_z
magnitude()
normalizar()
inverso()
produto_escalar()
produto_vetorial()
produto_triplo_escalar()
};
\end{lstlisting}

A classe vetor3d representará as coordenadas cartesianas tridimensionais no espaço.
Existem três componentes fundamentais:
eixo_X: Representa a distância do ponto em relação a uma origem de referência horizontalmente.
eixo_y: Representa verticalmente a distância do ponto até a origem.
eixo_z: É o eixo perpendicular ao eixo x e eixo y. Utilizado para dar profundidade.

O vetor também tem diversas operações associadas como soma, subtração,
multiplicação e divisão. Também possui métodos específicos como normalização,
magnitude, inverso e seus produtos como produto vetorial e escalar.

\begin{lstlisting}[frame=single,caption=Código de exemplo\label{codigo1}]
classe FormaGeometrica
{
tipo
recuperar_centroide()
calcular_suporte()
transladar(vetor v)
escalar(s)
rotacionar(orientação)
converter_string()
};
\end{lstlisting}

A classe de forma geométrica é uma base para as próximas implementações de
esferas e caixas.
Ela declara padrões que devem ser seguidos pelas classes filhas.

\begin{lstlisting}[frame=single,caption=Código de exemplo\label{codigo1}]
classe esfera estende FormaGeometrica
{
ponto_central
raio
};
\end{lstlisting}


A classe esfera tem 2 atributos principais que são:
Ponto_central: É um vetor que representa as coordenadas centrais da esfera.
Raio: Representa o quanto a esfera se estende em todas as direções a partir do
centro.

\begin{lstlisting}[frame=single,caption=Código de exemplo\label{codigo1}]
classe caixa  estende FormaGeometrica
{
mínimo
arestas
};
\end{lstlisting}

A classe caixa tem duas propriedades que também são importantes:
Mínimo: É um vetor que representa o canto inferior esquerdo da caixa.
Arestas: É o quanto as arestas se estendem nos 3 eixos a partir do canto
inferior esquerdo...

\begin{lstlisting}[frame=single,caption=Código de exemplo\label{codigo1}]
classe AABB
{
min
max
FormaGeometrica
translação(vetor v)
escala(s)
escala(vetor v)
computar_volume_delimitador()
};
\end{lstlisting}

Esta classe representa uma caixa alinhada ao eixo (AABB) Ela é utilizada para
envolver completamente uma forma geométrica, seja ela esfera, caixa ou
poliedro.
A classe possui 3 atributos principais:
min: É um vetor representando o canto inferior esquerdo do aabb...
Max: É um vetor representando o canto superior direito do aabb...
FormaGeometrica: É a forma geométrica que o aabb está contendo...
Também possui métodos para redimensionar e transladar o AABB.

\begin{lstlisting}[frame=single,caption=Código de exemplo\label{codigo1}]
classe colisao_info
{
ponto_de_colisao
normal_da_colisao
profundidade
objeto_1
objeto_2
};
\end{lstlisting}

A classe CollisionInfo é responsável por conter informações sobre a colisão que
foi detectada como ponto de colisão, normal, profundidade, e os objetos
envolvidos na colisão.

\begin{lstlisting}[frame=single,caption=Código de exemplo\label{codigo1}]
classe colisao
{
esfera_esfera(s1, s2, info)
esfera_caixa(s, c, info)
caixa_caixa(c1, c2, info)
esfera_poliedro(s, p, info)
caixa_poliedro(c, p, info)
poliedro_poliedro(p1, p2, info)
};
\end{lstlisting}

A classe de colisão irá conter os métodos específicos de colisão entre as
formas geométricas específicas como esferas x esferas, caixas x caixas e
poliedros x poliedros.

\begin{lstlisting}[frame=single,caption=Código de exemplo\label{codigo1}]
classe corpo_rigido_interface
    {
massa
restituição
nome
aabb
posição
    };
\end{lstlisting}

Esta interface define alguns atributos básicos que será necessário ao modelar o
objeto. Se precisarmos de atributos mais específicos, ou se quisermos expandir
nossa simulação futuramente, basta sub-classificarmos esta interface e realizar
as mudanças necessárias.

\begin{lstlisting}[frame=single,caption=Código de exemplo\label{codigo1}]
classe fase_ampla
{
escanear(lista_de_objetos, lista_de_possiveis_colisões)
};
\end{lstlisting}

Esta interface representa o nosso algoritmo de broadphase. Como existem
diversos algoritmos, optou-se de criar uma interface e realizar a implementação
separadamente.
O método scan recebe um vetor de corpos rígidos para verificar, e também uma
lista onde será armazenado informações sobre uma possível colisão...

\begin{lstlisting}[frame=single,caption=Código de exemplo\label{codigo1}]
classe fase_estreita
    {
detectar_colisões(lista_de_possiveis_colisões, lista_de_colisões)
    };
\end{lstlisting}

Esta classe implementa o algoritmo de fase estreita da mesma forma que a interface
de fase ampla. Este algoritmo recebe uma lista de possíveis colisões e sua
tarefa é aplicar os testes de detecção de colisão entre geometrias específicas
e avaliar se de fato, o par testado está colidindo.

\begin{lstlisting}[frame=single,caption=Código de exemplo\label{codigo1}]
classe solucionador_de_colisão
    {
resolver(lista_de_colisões)
resolver_par(objeto_1, objeto_2, info)
    };
\end{lstlisting}

Esta classe irá resolver as colisões detectadas. Ela age separando os corpos se
existir sobreposição, e aplicando impulso...

\begin{lstlisting}[frame=single,caption=Código de exemplo\label{codigo1}]
classe integrador_numérico
{
mover(lista_de_objetos, tempo)
mover_objeto(objeto, tempo)
};
\end{lstlisting}

 Esta classe implementa o algoritmo que fará os objetos se movimentarem na simulação. Para isso serão utilizadas equações das leis de Newton.

\begin{lstlisting}[frame=single,caption=Código de exemplo\label{codigo1}]
classe octree
{
raiz
adicionar_objeto(objeto)
remover_objeto(objeto)
fase_ampla(lista_de_objetos, lista_de_possíveis_colisões)
fase_ampla(objeto, lista_de_possíveis_colisões)
}

A classe octree é utilizada na fase ampla, mas como ela é relativamente cara de
se construir, ela é utilizada em objetos estáticos como obstáculos que representam a geografia do ambiente simulado.
A classe tem 2 métodos de fase ampla. O primeiro, recebe uma lista de todos objetos
que queremos que sejam testados, e a saída é uma lista com todas as colisões
detectadas. O segundo método, recebe um objeto e uma lista de saída de colisões detectadas.

\begin{lstlisting}[frame=single,caption=Código de exemplo\label{codigo1}]
class ambiente_de_simulação estende AABB
{
tempo_atual
gravidade
lista_de_objetos
fase_ampla
fase_estreita
solucionador_de_colisões
integrador_numérico
octree
adicionar_objeto(objeto)
remover_objeto(objeto)
passo_de_simulação(tempo)
};
\end{lstlisting}

Esta classe é responsável pela simulação do ambiente. Para isso recebe e manipula todos os objetos instanciados pelas classes anteriormente descritas. A classe permite adicionar e remover objetos, e também definir os algoritmos utilizados
para realizar as tarefas.
O método de atualização será chamado sempre que um passo na simulação for avançado.
Ele primeiramente movimentará os objetos, e então fará a fase ampla, fase
estreita, e resolverá as colisões...

Requisitos do Projeto

Compilador compatível com c++20 ou superior;
Compilador compatível com C2017 ou posterior;
wxwidgets para a criação das telas e controles como listas, botões e demais controles…
opengl para a renderização gráfica;
wxglade para construir a interface gráfica do usuário (GUI)
fmod e bass para indicações sonoras;

\subsection{Cronograma}
Agosto:
primeira quinzena:

Configuração do ambiente de desenvolvimento e modelagem das classes do
simulador.

segunda quinzena:
Revisão e testes das classes desenvolvidas
Registros e documentação parcial para composição do artigo final

Setembro
primeira quinzena:

Desenvolvimento da janela principal do aplicativo e seus diálogos.

segunda quinzena:
Desenvolvimento da animação gráfica dos objetos
Registro sobre o andamento do projeto para a composição do artigo final

Outubro
primeira quinzena
Refinamento de todo o projeto desenvolvido e realizar a integração dos componentes.

segunda quinzena
Finalização do protótipo para testes e correções de bugs
Registro do progresso para a composição do artigo final

Novembro
Primeira quinzena:

Testes do protótipo
Escrita do artigo final

Segunda quinzena:

Término do artigo final
Dezembro
primeira quinzena:
Finalização e entrega do artigo final
Defesa do Trabalho de conclusão de curso



